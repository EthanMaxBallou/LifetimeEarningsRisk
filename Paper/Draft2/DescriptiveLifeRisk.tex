\documentclass[12pt]{article}
\usepackage{setspace}
\usepackage{geometry}
\usepackage{hyperref}
\usepackage{multicol}
\usepackage{booktabs}
\usepackage{amssymb}
\usepackage{graphicx}
\usepackage{caption}
\usepackage{float}
\usepackage{setspace}
\usepackage{amsmath}
\usepackage{comment}




\geometry{margin=1in}
\bibliographystyle{plain}

\title{Predictors of Earnings Risk with Machine Learning}

\author{Ethan Ballou\thanks{University of Wisconsin - Milwaukee} \and Scott Drewianka$^{*}$}


\date{\today}

\begin{document}
\maketitle
\thispagestyle{empty}


%\date{\today}
%\pubMonth{Month}
%\pubYear{Year}
%\pubVolume{Vol}
%\pubIssue{Issue}
%\JEL{}
%\Keywords{}

\begin{abstract}
\begin{singlespace}


\noindent This paper looks at the determinants of lifetime earnings risk under a Restricted Income Profile (RIP) model using traditional and machine learning methods such as lasso and SHAP values. The paper builds on the work of Drewianka and Oberg (2025) which uses a moment condition approach derive a parameter that captures permanent income risk. The paper finds that education and age are important in explaining lifetime earnings risk. The paper also finds that macroeconomic variables such as probability of recession and real GDP growth are important and along with state controls may further imply a role of government policy. Finally, the paper finds that occupation controls are important while industry controls do not appear to play a strong role. 

\end{singlespace}
\end{abstract}


\vspace{1cm}

\noindent

\textbf{Keywords}: machine learning, restricted income profile, earnings instability, risk \\
\indent \textbf{JEL Codes}: D8, J0, D3\\




\clearpage
\setcounter{page}{1}

\begin{enumerate}
    \item NEED TO ADD INFO ABOUT WHICH DUMMIES WERE INCLUDED IN STEPWISE REGRESSIONS SINCE THEY ARE SUPPRESSED ON THE TABLE
    \item NEED TO ADD IN AND OCD Lasso, and of all of lasso for that matter
\end{enumerate}



\section{Introduction}

\section{Data and Model}

\section{Empirical Strategy}

\section{Results}

\section{Conclusion}




\begin{figure}[H]
    \centering
    \includegraphics[width=.7\textwidth]{/Users/ethanballou/Documents/GitHub/LifetimeEarningsRisk/Plots/histogram_gammaP_WEIGHTED.png}
    \caption{Distribution of Gamma}
\end{figure}


\begin{figure}[H]
    \centering
    \includegraphics[width=.7\textwidth]{/Users/ethanballou/Documents/GitHub/LifetimeEarningsRisk/Plots/scatter_age_gammaP_WEIGHTED.png}
    \caption{Scatterplot of Age vs. Gamma}
\end{figure}



\begin{table}[htbp]
    \centering
    \caption{Gamma Regressions: OLS Results}
    \label{tab:gamma_regressions}
    \resizebox{\textwidth}{!}{%
        {
\def\sym#1{\ifmmode^{#1}\else\(^{#1}\)\fi}
\begin{tabular}{l*{5}{c}}
\hline\hline
                    &\multicolumn{1}{c}{(1)}&\multicolumn{1}{c}{(2)}&\multicolumn{1}{c}{(3)}&\multicolumn{1}{c}{(4)}&\multicolumn{1}{c}{(5)}\\
                    &\multicolumn{1}{c}{Gamma}&\multicolumn{1}{c}{Gamma}&\multicolumn{1}{c}{Gamma}&\multicolumn{1}{c}{Gamma}&\multicolumn{1}{c}{Gamma}\\
\hline
Less than High School&     -0.0115\sym{***}&     -0.0107\sym{***}&     -0.0117\sym{***}&     -0.0109\sym{***}&     -0.0117\sym{***}\\
                    &   (0.00215)         &   (0.00232)         &   (0.00242)         &   (0.00253)         &   (0.00255)         \\
[1em]
High School Graduate&    -0.00918\sym{***}&    -0.00854\sym{***}&    -0.00902\sym{***}&    -0.00812\sym{***}&    -0.00850\sym{***}\\
                    &   (0.00144)         &   (0.00149)         &   (0.00158)         &   (0.00171)         &   (0.00173)         \\
[1em]
Some College        &    -0.00746\sym{***}&    -0.00681\sym{***}&    -0.00736\sym{***}&    -0.00674\sym{***}&    -0.00695\sym{***}\\
                    &   (0.00170)         &   (0.00172)         &   (0.00177)         &   (0.00184)         &   (0.00185)         \\
[1em]
Probability of Recession&   0.0000422         &     -0.0104         &  -0.0000953         &  -0.0000817         &  -0.0000292         \\
                    & (0.0000437)         &    (0.0179)         &     (51.52)         &     (51.50)         &     (51.49)         \\
[1em]
Real GDP growth rate&    0.000600         &    -0.00432         &  -0.0000211         &   -0.000186         &   -0.000626         \\
                    &  (0.000400)         &   (0.00325)         &     (3.981)         &     (3.979)         &     (3.979)         \\
[1em]
5-year moving average of AEP&   0.0000636\sym{**} &   0.0000629\sym{*}  &   0.0000374         &   0.0000766\sym{**} &   0.0000688\sym{*}  \\
                    & (0.0000295)         & (0.0000322)         & (0.0000340)         & (0.0000347)         & (0.0000356)         \\
[1em]
Out of Labor Force  &    -0.00430         &    -0.00410         &    -0.00471         &    -0.00461         &    -0.00474         \\
                    &   (0.00535)         &   (0.00536)         &   (0.00573)         &   (0.00572)         &   (0.00572)         \\
[1em]
Tenure              &  -0.0000438         &  -0.0000336         &   0.0000277         &  0.00000375         &   0.0000156         \\
                    & (0.0000888)         & (0.0000963)         & (0.0000981)         & (0.0000984)         & (0.0000987)         \\
[1em]
Age                 &     0.00592\sym{**} &     0.00650\sym{**} &     0.00679\sym{**} &     0.00679\sym{**} &     0.00671\sym{**} \\
                    &   (0.00269)         &   (0.00273)         &   (0.00274)         &   (0.00274)         &   (0.00274)         \\
[1em]
Age Squared         &   -0.000170\sym{***}&   -0.000193\sym{***}&   -0.000196\sym{***}&   -0.000197\sym{***}&   -0.000194\sym{***}\\
                    & (0.0000651)         & (0.0000659)         & (0.0000662)         & (0.0000662)         & (0.0000663)         \\
[1em]
Age Cubed           &  0.00000153\sym{***}&  0.00000172\sym{***}&  0.00000173\sym{***}&  0.00000173\sym{***}&  0.00000171\sym{***}\\
                    &(0.000000509)         &(0.000000515)         &(0.000000517)         &(0.000000518)         &(0.000000518)         \\
\hline
State FE            &          No         &         Yes         &         Yes         &         Yes         &         Yes         \\
Year FE             &          No         &         Yes         &         Yes         &         Yes         &         Yes         \\
Race FE             &          No         &         Yes         &         Yes         &         Yes         &         Yes         \\
Cohort FE           &          No         &         Yes         &         Yes         &         Yes         &         Yes         \\
Occupation FE       &          No         &          No         &          No         &         Yes         &         Yes         \\
Industry FE         &          No         &          No         &         Yes         &          No         &         Yes         \\
R-squared           &       0.001         &       0.002         &       0.003         &       0.005         &       0.006         \\
N                   &       82357         &       82333         &       81556         &       81556         &       81556         \\
\hline\hline
\multicolumn{6}{l}{\footnotesize Standard errors in parentheses}\\
\multicolumn{6}{l}{\footnotesize \sym{*} \(p<0.10\), \sym{**} \(p<0.05\), \sym{***} \(p<0.01\)}\\
\end{tabular}
}

    }
\end{table}


\begin{table}[htbp]
    \centering
    \caption{Gamma Regressions: Stepwise Selection Results}
    \label{tab:gamma_stepwise}
    \resizebox{\textwidth}{!}{%
        {
\def\sym#1{\ifmmode^{#1}\else\(^{#1}\)\fi}
\begin{tabular}{l*{5}{c}}
\hline\hline
                    &\multicolumn{1}{c}{(1)}&\multicolumn{1}{c}{(2)}&\multicolumn{1}{c}{(3)}&\multicolumn{1}{c}{(4)}&\multicolumn{1}{c}{(5)}\\
                    &\multicolumn{1}{c}{gammaP\_WEIGHTED}&\multicolumn{1}{c}{gammaP\_WEIGHTED}&\multicolumn{1}{c}{gammaP\_WEIGHTED}&\multicolumn{1}{c}{gammaP\_WEIGHTED}&\multicolumn{1}{c}{gammaP\_WEIGHTED}\\
\hline
EDU1                &    -0.00768\sym{*}  &    -0.00769\sym{*}  &                     &                     &                     \\
                    &   (0.00363)         &   (0.00364)         &                     &                     &                     \\
[1em]
EDU2                &    -0.00489\sym{*}  &    -0.00494\sym{*}  &                     &                     &                     \\
                    &   (0.00212)         &   (0.00212)         &                     &                     &                     \\
[1em]
currentage cubed    & 0.000000364\sym{*}  & 0.000000367\sym{*}  & 0.000000449\sym{**} & 0.000000558\sym{***}& 0.000000476\sym{**} \\
                    &(0.000000145)         &(0.000000145)         &(0.000000157)         &(0.000000162)         &(0.000000158)         \\
[1em]
currentage squared  &  -0.0000222\sym{*}  &  -0.0000225\sym{*}  &  -0.0000287\sym{**} &  -0.0000352\sym{**} &  -0.0000304\sym{**} \\
                    &(0.00000985)         &(0.00000985)         & (0.0000108)         & (0.0000111)         & (0.0000109)         \\
[1em]
twoind==Agric.,Forestry&                     &                     &     0.00842         &                     &     0.00823         \\
                    &                     &                     &   (0.00809)         &                     &   (0.00811)         \\
[1em]
twoind==Fisheries   &                     &                     &      0.0586         &                     &      0.0585         \\
                    &                     &                     &    (0.0390)         &                     &    (0.0391)         \\
[1em]
twoind==Energy/Water&                     &                     &    -0.00287         &                     &    -0.00282         \\
                    &                     &                     &   (0.00645)         &                     &   (0.00649)         \\
[1em]
twoind==Mining      &                     &                     &      0.0229         &                     &      0.0251\sym{*}  \\
                    &                     &                     &    (0.0120)         &                     &    (0.0121)         \\
[1em]
twoind==Chemicals   &                     &                     &    -0.00639         &                     &    -0.00653         \\
                    &                     &                     &   (0.00835)         &                     &   (0.00836)         \\
[1em]
twoind==Synthetics  &                     &                     &     0.00519         &                     &     0.00511         \\
                    &                     &                     &    (0.0103)         &                     &    (0.0104)         \\
[1em]
twoind==Earth/Clay/Stone&                     &                     &    -0.00334         &                     &    0.000439         \\
                    &                     &                     &    (0.0125)         &                     &    (0.0127)         \\
[1em]
twoind==Iron/Steel  &                     &                     &     0.00161         &                     &     0.00127         \\
                    &                     &                     &   (0.00701)         &                     &   (0.00706)         \\
[1em]
twoind==Mechanical Eng&                     &                     &    -0.00333         &                     &    -0.00352         \\
                    &                     &                     &   (0.00478)         &                     &   (0.00480)         \\
[1em]
twoind==Electrical Eng&                     &                     &     0.00223         &                     &     0.00250         \\
                    &                     &                     &   (0.00617)         &                     &   (0.00626)         \\
[1em]
twoind==Wood/Paper/Print&                     &                     &     0.00387         &                     &     0.00391         \\
                    &                     &                     &   (0.00593)         &                     &   (0.00595)         \\
[1em]
twoind==Clothing/Text&                     &                     &    -0.00593         &                     &    -0.00607         \\
                    &                     &                     &   (0.00839)         &                     &   (0.00841)         \\
[1em]
twoind==Food Industry&                     &                     &     0.00551         &                     &     0.00314         \\
                    &                     &                     &   (0.00830)         &                     &   (0.00833)         \\
[1em]
twoind==Construction&                     &                     &      0.0233\sym{**} &                     &      0.0232\sym{**} \\
                    &                     &                     &   (0.00847)         &                     &   (0.00849)         \\
[1em]
twoind==Constr. Relate&                     &                     &      0.0102\sym{*}  &                     &      0.0105\sym{*}  \\
                    &                     &                     &   (0.00493)         &                     &   (0.00497)         \\
[1em]
twoind==Wholesale   &                     &                     &    -0.00739         &                     &    -0.00658         \\
                    &                     &                     &   (0.00579)         &                     &   (0.00583)         \\
[1em]
twoind==Retail      &                     &                     &      0.0106\sym{*}  &                     &      0.0105\sym{*}  \\
                    &                     &                     &   (0.00465)         &                     &   (0.00469)         \\
[1em]
twoind==Train System&                     &                     &    -0.00177         &                     &    -0.00188         \\
                    &                     &                     &    (0.0115)         &                     &    (0.0116)         \\
[1em]
twoind==Postal System&                     &                     &    0.000549         &                     &    0.000450         \\
                    &                     &                     &   (0.00574)         &                     &   (0.00576)         \\
[1em]
twoind==Other Trans.&                     &                     &     0.00490         &                     &     0.00487         \\
                    &                     &                     &   (0.00526)         &                     &   (0.00528)         \\
[1em]
twoind==Financial Inst&                     &                     &    -0.00293         &                     &    -0.00433         \\
                    &                     &                     &   (0.00917)         &                     &   (0.00927)         \\
[1em]
twoind==Insurance   &                     &                     &     0.00283         &                     &     0.00156         \\
                    &                     &                     &   (0.00856)         &                     &   (0.00858)         \\
[1em]
twoind==Restaurants &                     &                     &     -0.0458\sym{**} &                     &     -0.0455\sym{**} \\
                    &                     &                     &    (0.0144)         &                     &    (0.0149)         \\
[1em]
twoind==Service Indust&                     &                     &  0.00000869         &                     &   -0.000147         \\
                    &                     &                     &    (0.0100)         &                     &    (0.0101)         \\
[1em]
twoind==Educ./Sport &                     &                     &    -0.00577         &                     &    -0.00595         \\
                    &                     &                     &   (0.00573)         &                     &   (0.00578)         \\
[1em]
twoind==Health Service&                     &                     &   -0.000594         &                     &   -0.000828         \\
                    &                     &                     &   (0.00623)         &                     &   (0.00627)         \\
[1em]
twoind==Legal Services&                     &                     &      0.0243\sym{**} &                     &      0.0240\sym{**} \\
                    &                     &                     &   (0.00771)         &                     &   (0.00778)         \\
[1em]
twoind==Other Services&                     &                     &     0.00762         &                     &     0.00759         \\
                    &                     &                     &   (0.00491)         &                     &   (0.00495)         \\
[1em]
twoind==Volunt./Church&                     &                     &    -0.00936         &                     &     -0.0103         \\
                    &                     &                     &   (0.00935)         &                     &   (0.00947)         \\
[1em]
twoind==Priv. Househld&                     &                     &     -0.0193         &                     &     -0.0196         \\
                    &                     &                     &     (0.151)         &                     &     (0.151)         \\
[1em]
Constant            &      0.0304\sym{***}&      0.0306\sym{***}&      0.0292\sym{***}&      0.0349\sym{***}&      0.0302\sym{***}\\
                    &   (0.00619)         &   (0.00619)         &   (0.00726)         &   (0.00710)         &   (0.00732)         \\
\hline
State FE            &          No         &         Yes         &         Yes         &         Yes         &         Yes         \\
Year FE             &          No         &         Yes         &         Yes         &         Yes         &         Yes         \\
Race FE             &          No         &         Yes         &         Yes         &         Yes         &         Yes         \\
Cohort FE           &          No         &         Yes         &         Yes         &         Yes         &         Yes         \\
Occupation FE       &          No         &          No         &          No         &         Yes         &         Yes         \\
Industry FE         &          No         &          No         &         Yes         &          No         &         Yes         \\
R-squared           &       0.001         &       0.001         &       0.004         &       0.001         &       0.004         \\
N                   &       25920         &       25913         &       20954         &       20823         &       20746         \\
\hline\hline
\multicolumn{6}{l}{\footnotesize Standard errors in parentheses}\\
\multicolumn{6}{l}{\footnotesize \sym{*} \(p<0.05\), \sym{**} \(p<0.01\), \sym{***} \(p<0.001\)}\\
\end{tabular}
}

    }
\end{table}






\begin{figure}[H]
    \centering
    \includegraphics[width=.7\textwidth]{/Users/ethanballou/Documents/GitHub/LifetimeEarningsRisk/Plots/histogram_alphaP_WEIGHTED.png}
    \caption{Distribution of Alpha}
\end{figure}


\begin{figure}[H]
    \centering
    \includegraphics[width=.7\textwidth]{/Users/ethanballou/Documents/GitHub/LifetimeEarningsRisk/Plots/scatter_age_alphaP_WEIGHTED.png}
    \caption{Scatterplot of Age vs. Alpha}
\end{figure}



\begin{table}[htbp]
    \centering
    \caption{Alpha Regressions: OLS Results}
    \label{tab:alpha_regressions}
    \resizebox{\textwidth}{!}{%
        {
\def\sym#1{\ifmmode^{#1}\else\(^{#1}\)\fi}
\begin{tabular}{l*{5}{c}}
\hline\hline
                    &\multicolumn{1}{c}{(1)}&\multicolumn{1}{c}{(2)}&\multicolumn{1}{c}{(3)}&\multicolumn{1}{c}{(4)}&\multicolumn{1}{c}{(5)}\\
                    &\multicolumn{1}{c}{alphaP\_WEIGHTED}&\multicolumn{1}{c}{alphaP\_WEIGHTED}&\multicolumn{1}{c}{alphaP\_WEIGHTED}&\multicolumn{1}{c}{alphaP\_WEIGHTED}&\multicolumn{1}{c}{alphaP\_WEIGHTED}\\
\hline
EDU1                &     -0.0442\sym{***}&     -0.0342\sym{***}&     -0.0299\sym{**} &     -0.0285\sym{*}  &     -0.0271\sym{*}  \\
                    &   (0.00889)         &   (0.00954)         &    (0.0108)         &    (0.0114)         &    (0.0115)         \\
[1em]
EDU2                &     -0.0387\sym{***}&     -0.0364\sym{***}&     -0.0341\sym{***}&     -0.0301\sym{***}&     -0.0277\sym{***}\\
                    &   (0.00580)         &   (0.00600)         &   (0.00664)         &   (0.00736)         &   (0.00743)         \\
[1em]
EDU3                &     -0.0234\sym{***}&     -0.0242\sym{***}&     -0.0184\sym{*}  &     -0.0162\sym{*}  &     -0.0146         \\
                    &   (0.00666)         &   (0.00681)         &   (0.00732)         &   (0.00774)         &   (0.00778)         \\
[1em]
Probability of Recession&  -0.0000102         &     -0.0449         &     0.00151         &       0.255         &       0.245         \\
                    &  (0.000185)         &     (0.167)         &     (664.2)         &     (0.216)         &     (0.216)         \\
[1em]
Real GDP growth rate&    -0.00108         &    -0.00306         &      0.0169         &      0.0130         &      0.0126         \\
                    &   (0.00158)         &    (0.0135)         &     (5.775)         &    (0.0175)         &    (0.0176)         \\
[1em]
5-year moving average of AEP&     0.00142\sym{***}&     0.00153\sym{***}&     0.00129\sym{***}&     0.00148\sym{***}&     0.00144\sym{***}\\
                    &  (0.000119)         &  (0.000129)         &  (0.000144)         &  (0.000149)         &  (0.000154)         \\
[1em]
veteran             &     -0.0124         &     -0.0122         &    -0.00408         &   -0.000294         &    -0.00321         \\
                    &   (0.00920)         &   (0.00941)         &    (0.0146)         &    (0.0150)         &    (0.0152)         \\
[1em]
=1 if out of labor force (inc retired)&       0.122\sym{***}&       0.123\sym{***}&      0.0984\sym{***}&      0.0891\sym{**} &      0.0923\sym{**} \\
                    &    (0.0165)         &    (0.0165)         &    (0.0289)         &    (0.0291)         &    (0.0291)         \\
[1em]
tenure              &   -0.000825\sym{*}  &  -0.0000632         & -0.00000483         &   -0.000294         &   -0.000244         \\
                    &  (0.000330)         &  (0.000355)         &  (0.000365)         &  (0.000373)         &  (0.000372)         \\
[1em]
age (consistent across years)&      0.0158         &      0.0186         &     0.00916         &      0.0162         &      0.0113         \\
                    &   (0.00985)         &    (0.0101)         &    (0.0115)         &    (0.0116)         &    (0.0116)         \\
[1em]
currentage squared  &   -0.000433         &   -0.000533\sym{*}  &   -0.000283         &   -0.000446         &   -0.000321         \\
                    &  (0.000229)         &  (0.000233)         &  (0.000263)         &  (0.000266)         &  (0.000266)         \\
[1em]
currentage cubed    &  0.00000394\sym{*}  &  0.00000477\sym{**} &  0.00000278         &  0.00000403\sym{*}  &  0.00000299         \\
                    &(0.00000171)         &(0.00000175)         &(0.00000195)         &(0.00000198)         &(0.00000197)         \\
\hline
State FE            &          No         &         Yes         &         Yes         &         Yes         &         Yes         \\
Year FE             &          No         &         Yes         &         Yes         &         Yes         &         Yes         \\
Race FE             &          No         &         Yes         &         Yes         &         Yes         &         Yes         \\
Cohort FE           &          No         &         Yes         &         Yes         &         Yes         &         Yes         \\
Occupation FE       &          No         &          No         &          No         &         Yes         &         Yes         \\
Industry FE         &          No         &          No         &         Yes         &          No         &         Yes         \\
R-squared           &       0.013         &       0.019         &       0.029         &       0.034         &       0.037         \\
N                   &       32567         &       32556         &       25592         &       25426         &       25329         \\
\hline\hline
\multicolumn{6}{l}{\footnotesize Standard errors in parentheses}\\
\multicolumn{6}{l}{\footnotesize \sym{*} \(p<0.05\), \sym{**} \(p<0.01\), \sym{***} \(p<0.001\)}\\
\end{tabular}
}

    }
\end{table}

\begin{table}[htbp]
    \centering
    \caption{Alpha Regressions: Stepwise Selection Results}
    \label{tab:alpha_stepwise}
    \resizebox{\textwidth}{!}{%
        {
\def\sym#1{\ifmmode^{#1}\else\(^{#1}\)\fi}
\begin{tabular}{l*{5}{c}}
\hline\hline
                    &\multicolumn{1}{c}{(1)}&\multicolumn{1}{c}{(2)}&\multicolumn{1}{c}{(3)}&\multicolumn{1}{c}{(4)}&\multicolumn{1}{c}{(5)}\\
                    &\multicolumn{1}{c}{Alpha}&\multicolumn{1}{c}{Alpha}&\multicolumn{1}{c}{Alpha}&\multicolumn{1}{c}{Alpha}&\multicolumn{1}{c}{Alpha}\\
\hline
Less than High School&     -0.0456\sym{***}&     -0.0350\sym{***}&     -0.0308\sym{**} &     -0.0296\sym{**} &                     \\
                    &   (0.00877)         &   (0.00947)         &    (0.0107)         &    (0.0113)         &                     \\
[1em]
High School Graduate&     -0.0395\sym{***}&     -0.0374\sym{***}&     -0.0346\sym{***}&     -0.0305\sym{***}&     -0.0156\sym{**} \\
                    &   (0.00578)         &   (0.00596)         &   (0.00657)         &   (0.00729)         &   (0.00538)         \\
[1em]
Some College        &     -0.0237\sym{***}&     -0.0255\sym{***}&     -0.0185\sym{*}  &     -0.0158\sym{*}  &                     \\
                    &   (0.00665)         &   (0.00677)         &   (0.00725)         &   (0.00769)         &                     \\
[1em]
Age Cubed           &  0.00000131\sym{***}&  0.00000176\sym{***}&  0.00000119\sym{***}&  0.00000129\sym{***}&  0.00000112\sym{**} \\
                    &(0.000000303)         &(0.000000329)         &(0.000000338)         &(0.000000344)         &(0.000000343)         \\
[1em]
Age Squared         &  -0.0000734\sym{***}&   -0.000119\sym{***}&  -0.0000686\sym{**} &  -0.0000752\sym{**} &  -0.0000653\sym{**} \\
                    & (0.0000212)         & (0.0000248)         & (0.0000239)         & (0.0000243)         & (0.0000242)         \\
[1em]
5-year moving average of AEP&     0.00146\sym{***}&     0.00152\sym{***}&     0.00127\sym{***}&     0.00149\sym{***}&     0.00137\sym{***}\\
                    &  (0.000115)         &  (0.000121)         &  (0.000139)         &  (0.000144)         &  (0.000146)         \\
[1em]
Tenure              &   -0.000824\sym{*}  &                     &                     &                     &                     \\
                    &  (0.000329)         &                     &                     &                     &                     \\
[1em]
Out of Labor Force  &       0.124\sym{***}&       0.126\sym{***}&       0.101\sym{***}&      0.0942\sym{**} &      0.0973\sym{***}\\
                    &    (0.0164)         &    (0.0164)         &    (0.0287)         &    (0.0289)         &    (0.0289)         \\
[1em]
Constant            &      0.0774\sym{***}&       0.154\sym{**} &       0.105         &       0.102         &      0.0986         \\
                    &    (0.0158)         &    (0.0590)         &    (0.0585)         &    (0.0591)         &    (0.0587)         \\
\hline
State FE            &          No         &         Yes         &         Yes         &         Yes         &         Yes         \\
Year FE             &          No         &         Yes         &         Yes         &         Yes         &         Yes         \\
Race FE             &          No         &         Yes         &         Yes         &         Yes         &         Yes         \\
Cohort FE           &          No         &         Yes         &         Yes         &         Yes         &         Yes         \\
Occupation FE       &          No         &          No         &          No         &         Yes         &         Yes         \\
Industry FE         &          No         &          No         &         Yes         &          No         &         Yes         \\
R-squared           &       0.013         &       0.019         &       0.028         &       0.034         &       0.036         \\
N                   &       32567         &       32556         &       25592         &       25426         &       25329         \\
\hline\hline
\multicolumn{6}{l}{\footnotesize Standard errors in parentheses}\\
\multicolumn{6}{l}{\footnotesize \sym{*} \(p<0.05\), \sym{**} \(p<0.01\), \sym{***} \(p<0.001\)}\\
\end{tabular}
}

    }
\end{table}


\end{document}


