\documentclass[12pt]{article}
\usepackage{setspace}
\usepackage{geometry}
\usepackage{hyperref}
\usepackage{multicol}
\usepackage{booktabs}
\usepackage{amssymb}
\usepackage{graphicx}
\usepackage{caption}
\usepackage{float}
\usepackage{setspace}
\usepackage{amsmath}
\usepackage{comment}




\geometry{margin=1in}
\bibliographystyle{plain}

\title{Predictors of Earnings Risk with Machine Learning}

\author{Ethan Ballou\thanks{University of Wisconsin - Milwaukee}}


\date{\today}

\begin{document}
\maketitle
\thispagestyle{empty}


%\date{\today}
%\pubMonth{Month}
%\pubYear{Year}
%\pubVolume{Vol}
%\pubIssue{Issue}
%\JEL{}
%\Keywords{}

\begin{abstract}
\begin{singlespace}


\noindent This paper looks at the determinants of lifetime earnings risk under a Restricted Income Profile (RIP) model using traditional and machine learning methods such as lasso and SHAP values. The paper builds on the work of Drewianka and Oberg (2025) which uses a moment condition approach derive a parameter that captures permanent income risk. The paper finds that education and age are important in explaining lifetime earnings risk. The paper also finds that macroeconomic variables such as probability of recession and real GDP growth are important and along with state controls may further imply a role of government policy. Finally, the paper finds that occupation controls are important while industry controls do not appear to play a strong role. 

\end{singlespace}
\end{abstract}


\vspace{1cm}

\noindent

\textbf{Keywords}: machine learning, restricted income profile, earnings instability, risk \\
\indent \textbf{JEL Codes}: D8, J0, D3\\




\clearpage
\setcounter{page}{1}












\end{document}


